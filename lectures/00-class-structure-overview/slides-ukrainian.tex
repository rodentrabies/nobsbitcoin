\documentclass{beamer}


\usepackage{amsmath}
\usepackage[style=alphabetic,url=true]{biblatex}
\usepackage{environ}
\usepackage{geometry}
\usepackage{graphicx}
\usepackage{tikz}
\usepackage[T2A]{fontenc}
\usepackage[utf8]{inputenc}
\usepackage{listings}
\usepackage{cancel}
\usepackage{soul}


% \usetheme{Bergen}

\usecolortheme{beaver}

\setbeamertemplate{itemize item}[circle]
\setbeamertemplate{itemize subitem}{--}
\addtobeamertemplate{navigation symbols}{}{
  \usebeamerfont{footline}%
  \usebeamercolor[fg]{footline}%
  \hspace{1em}%
  \insertframenumber/\inserttotalframenumber
}
\graphicspath{ {./graphics/} }


\title{
  Біткоїн та криптовалютні технології \\
  Лекція 0: Огляд структури курсу
}

\author{Юрій Жикін}
\date{17 лютого, 2025}

\begin{document}

\frame{\titlepage}

\begin{frame}
  \frametitle{Контакти}
  \begin{itemize}
  \item @rodentrabies в Telegram
  \item https://github.com/rodentrabies
  \end{itemize}
\end{frame}

\begin{frame}
  \frametitle{Структура курсу 1/2}
  \begin{itemize}
  \item Історія та економіка Біткоїна
    \begin{itemize}
    \item Економічні концепції та властивості грошей
    \item Комп'ютерна криптографія та шифропанк-рух
    \item Винахід та інновація Біткоїна
    \end{itemize}
  \item ``Крипто'' означає ``криптографія''
    \begin{itemize}
    \item Основи криптографія
    \item Хеш-функції
    \item Криптографія з відкритим ключем
    \item Еліптичні криві
    \item Криптографічні підписи
    \end{itemize}
  \item Модель даних ланцюга Біткоїна
    \begin{itemize}
    \item Транзакції
    \item Транзакційні входи та виходи
    \item Блоки
    \item Ланцюг блоків та ``доказ виконаної роботи''
    \end{itemize}
  \item[] ...
  \end{itemize}
\end{frame}

\begin{frame}
  \frametitle{Структура курсу 2/2}
  \begin{itemize}
  \item[] ...
  \item Детальний погляд на транзакції у Біткоїні
    \begin{itemize}
    \item Транзакційні скрипти
    \item Валідація транзакцій
    \item ``Біткоїн-гаманець''
    \end{itemize}
  \item Мережа Біткоїн
    \begin{itemize}
    \item Однорангова мережева архітектура
    \item Басейн транзакцій та процес ``майнінгу''
    \item Параметри та динаміка мережі
    \end{itemize}
  \item Протоколи другого рівня
    \begin{itemize}
    \item Проблема пропускної здатності
    \item Мережі платіжних каналів
    \item Незамінні токени
    \end{itemize}
  \item Інші криптовалютні системи
    \begin{itemize}
    \item Ethereum: максимальна гнучкість
    \item Monero: максимальна приватність
    \end{itemize}
  \end{itemize}
\end{frame}

\begin{frame}
  \frametitle{Контроль}
  \begin{itemize}
  \item 2 лабораторні роботи (перший рік)
    \begin{itemize}
    \item Встановлення та налаштування програмного забезпечення Біткоїн-вузла
    \item Взаємодія з Біткоїн-мережею з командного рядка
    \end{itemize}
  \item 1 лабораторна робота (другий рік)
    \begin{itemize}
    \item Написання набору функцій для парсингу даних ланцюга блоків
    \end{itemize}
  \item Підсумковий тест
    \begin{itemize}
    \item 20/100 випадкових питань з одним/декількома варіантами відповіді
    \end{itemize}
  \end{itemize}
\end{frame}

\end{document}

%%% Local Variables:
%%% mode: latex
%%% TeX-master: t
%%% End:
