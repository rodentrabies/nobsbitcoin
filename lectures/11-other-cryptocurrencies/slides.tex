\documentclass{beamer}


\usepackage{amsmath}
\usepackage[style=alphabetic,url=true]{biblatex}
\usepackage{environ}
\usepackage{geometry}
\usepackage{graphicx}
\usepackage{tikz}
\usepackage[T2A]{fontenc}
\usepackage[utf8]{inputenc}
\usepackage[cache=false]{minted}
\usepackage{amsmath}
\usepackage{amsfonts}
\usepackage{amssymb}
\usepackage{calrsfs}
\usepackage{animate}
\usepackage{xmpmulti}


% \usetheme{Bergen}

\usecolortheme{beaver}

\setbeamertemplate{itemize item}[circle]
\setbeamertemplate{itemize subitem}{--}
\addtobeamertemplate{navigation symbols}{}{
  \usebeamerfont{footline}%
  \usebeamercolor[fg]{footline}%
  \hspace{1em}%
  \insertframenumber/\inserttotalframenumber
}
\graphicspath{ {./graphics/} }
\setminted[Python]{
  fontsize=\tiny
}
\setminted[Lisp]{
  fontsize=\tiny
}
\BeforeBeginEnvironment{minted}{\medskip}
\AfterEndEnvironment{minted}{\medskip}
\usetikzlibrary{matrix}
\tikzset{
  stack/.style={
    matrix of nodes,
    nodes={
      fill=lightgray,draw,text=black,font=\sffamily\bfseries,
      text height=11pt,text depth=3pt,baseline=center, minimum width=1cm
    },
    column sep=-\pgflinewidth/2
  }
}

\title{
  Bitcoin and Cryptocurrency Technologies \\
  Lecture 11: Other Cryptocurrencies
}

\author{Yuri Zhykin}
\date{Apr 28, 2025}

\begin{document}

\frame{\titlepage}

\begin{frame}
  \frametitle{Problems and Solutions}
  \begin{itemize}
  \item As of 2021, there is more than 4000 cryptocurrencies, most of which
    have no useful application.
  \item Various cryptocurrencies claim to ``solve'' one or several problems of
    Bitcoin protocol, such as:
    \begin{itemize}
    \item ``small'' blocks, ``long'' periods between blocks,
    \item ``weak'' (inflexible) scripting system,
    \item mining centralization,
    \item energy consumption,
    \item \textbf{transaction privacy}.
    \end{itemize}
  \item Bitcoin, unlike most other cryptocurrency systems, does not have a
    \textbf{central governing body}.
  \end{itemize}
\end{frame}

\begin{frame}
  \frametitle{Scripting System 1/3}
  \begin{itemize}
  \item One of the first ideas of the ``next generation'' cryptocurrencies: make
    the transaction validation language Turing-complete to allow arbitrary
    ``smart contracts''.
  \item A \textbf{smart contract} is a \textbf{self-executing} contract with the
    terms of the agreement between the parties specified as code.
  \end{itemize}
\end{frame}

\begin{frame}
  \frametitle{Scripting System 2/3}
  \begin{itemize}
  \item Bitcoin Script is a \textbf{simple stack-based Turing-incomplete}
    language by design.
  \item Bitcoin's approach:
    \begin{itemize}
    \item simple language is easier to analyze for incorrect behavior;
    \item Turing-incompleteness prevents infinite loops and undefined behavior;
    \item most ``smart contracts'' are simple enough to be expressed in Bitcoin
      Script;
    \item ``smart contracts'' that cannot be expressed in Bitcoin Script,
      can/should be implemented as second-layer protocols.
    \end{itemize}
  \item In fact, Bitcoin Script is considered by some to \textbf{too flexible}
    (\textbf{Miniscript}, \textbf{Simplicity}).
  \end{itemize}
\end{frame}

\begin{frame}
  \frametitle{Scripting System 3/3}
  \begin{itemize}
  \item Main types of smart contracts on Ethereum platform:
    \begin{itemize}
    \item \textbf{fungible} tokens similar to stocks (ICOs based on ERC-20
      protocol, 2017);
    \item \textbf{non-fungible} (collectible) tokens like CryptoKitties (2017)
      or NFTs (2020);
    \item digital identity management.
    \end{itemize}
  \item Both fungible and non-fungible tokens can be natively represented on
    Bitcoin via \textbf{``coloring''} - UTXO is ``marked'' and considered a new
    token - multiple projects around 2015.
  \item All of the above can be implemented on top of Bitcoin using ideas like
    \textbf{client side validation} (e.g. \textbf{RGB protocol} on the Lightning
    Network), providing both \textbf{contract flexibility} and
    \textbf{scalability}.
  \end{itemize}
\end{frame}

\begin{frame}
  \frametitle{Mining Centralization 1/2}
  \begin{itemize}
  \item Bitcoin's Proof-of-Work system uses simple SHA-256d (double SHA-256)
    algorithm.
  \item SHA-256d can be easily implemented in specialized hardware
    (\textbf{ASICs} - \textbf{Application Specific Integrated Circuits}.
  \item As of 2022, Bitcoin is mined solely on ASICs, as any generic purpose
    hardware is way to slow to be profitable.
  \item Bitcoin ASICs are mostly produced in China, which, combined with the low
    electricity consts, resulted in an accumulation of significant amounts of
    \textbf{hash-power} in mainland China in 2014-2017.
  \item Some people within cryptocurrency community consider ASIC resistance to
    be an important characteristic of the ``next generation'' cryptocurrency
    system.
  \end{itemize}
\end{frame}

\begin{frame}
  \frametitle{Mining Centralization 1/2}
  \begin{itemize}
  \item Bitcoin's point of view:
    \begin{itemize}
    \item ASICs are slowly approaching theoretical efficiency limits, so the
      incentive to install them close to the manufacturer (i.e. in China) is
      diminishing;
    \item a decentralization tendency has been observed in the last several
      years - more mining operations are being set up in other countries,
    \item that said, there are discussions of a potential change of the PoW
      algorithm in a distant future - this will most definitely require a
      hard-fork.
    \end{itemize}
  \end{itemize}
\end{frame}

\begin{frame}
  \frametitle{Energy Consumption}
  \begin{itemize}
  \item Lately, Bitcoin is being aggressively criticized by the mainstream
    media for its total energy consumption.
  \item Most of the claims on Bitcoin's energy consumption show little to no
    understanding of both Bitcoin and electricity usage (discussed in great
    detail by Nic Carter on Twitter):
    \begin{itemize}
    \item excess electricity generated from renewables cannot be stored;
    \item no evidence of Bitcoin mining operations causing electrical grid
      outages in their respective areas, which indicates that only the excess
      electricity is being consumed;
    \item miners are incentivized to put their equipment close to cheap excess
      electricity sources;
    \item Bitcoin mining will optimize inefficient renewable energy production
      and incentivize \textbf{nuclear energy}.
    \end{itemize}
  \item \textbf{Does Bitcoin's value proposition outweigh the costs of
      consumed electricity?}
  \end{itemize}
\end{frame}

\begin{frame}
  \frametitle{Proof of Stake}
  \begin{itemize}
  \item \textbf{Proof of stake} (PoS) protocols are a class of consensus
    mechanisms for blockchains that work by selecting validators in proportion
    to their quantity of holdings in the associated cryptocurrency.
  \item The first functioning use of PoS for cryptocurrency was Peercoin in
    2012.
  \item Proof of stake violates the energy-based interpretation of decentralized
    consensus: \textbf{Bitcoin blocks are secured by large amounts of energy
      needed to generate one}.
  \item ``Proof of Stake is why Proof of Work was invented.'' - @notgrubles on
    Twitter.
  \end{itemize}
\end{frame}

\begin{frame}
  \frametitle{Transaction Privacy 1/2}
  \begin{itemize}
  \item Bitcoin block chain data is transparent: for every ``chunk'' of bitcoin
    created by a coinbase transaction, the whole transaction history can be
    followed by simply looking at the chain data.
  \item Bitcoin addresses are \textbf{pseudonymous}: outputs to the same address
    can be easily tied together, \textbf{but one should not reuse addresses}.
  \item As a result, Bitcoin chain is susceptible to \textbf{chain analysis}
    that in some cases allows to establish the sender and the recipient of the
    bitcoin (the amount is publicly visible directly).
  \item This might complicate Bitcoin's use as a currency for daily payments,
    where transaction privacy is essential (salary, medication bills, etc).
  \end{itemize}
\end{frame}

\begin{frame}
  \frametitle{Transaction Privacy 2/2}
  \begin{itemize}
  \item Lightning Network significantly improves transaction privacy by moving
    day-to-day transactions off chain.
  \item Channel open and channel close transactions are still visible on chain -
    \textbf{taproot} partially fixes this and is a first step towards a general
    solution to the problem.
  \item Proposals to implement privacy features on Bitcoin \textbf{sidechains}.
  \item Proposals to implement privacy features directly on Bitcoin mainnet -
    unlikely to happen in the near future, but will probably be implemented
    eventually in one way or another.
  \end{itemize}
\end{frame}

\begin{frame}
  \frametitle{Case Study: Ethereum 1/2}
  \begin{itemize}
  \item Created by Vitalik Buterin - a Russian-Canadian programmer and an active
    member of Bitcoin community in the early days.
  \item Launched in 2014.
  \item Main difference from Bitcoin: Turing-complete smart contract language
    Solidity with JavaScript-like syntax.
  \item In order to avoid the infinite loops and transaction spam, Ethereum's
    fee system requires users to pay for operations executed by their smart
    contracts - this is called \textbf{paying for gas}.
  \end{itemize}
\end{frame}

\begin{frame}
  \frametitle{Case Study: Ethereum 2/2}
  \begin{itemize}
  \item Turing-completeness of the Ethereum was the reason behind the infamous
    ``DAO hack'' - loss of a large amount of funds in 2016 that caused
    \textbf{the Ethereum governing body to revert a transaction via a hard
      fork}.
  \item An investigation in 2019 \textbf{indicated} that 60\% of Ethereum nodes
    were running in the cloud on centralized platforms like AWS (claim by AWS
    confirms that 25\% of Ethereum infrastructure runs there in 2022).
  \item Ethereum's scalability problem is even worse than that of Bitcoin: full
    node is 700 Gb, archive node is approximately 10 Tb of data (Bitcoin
    blockchain for comparison is 380 Gb, and it's twice as old).
  \item Ethereum network is determined to move to a Proof-of-Stake system.
  \end{itemize}
\end{frame}

\begin{frame}
  \frametitle{Case Study: Monero 1/2}
  \begin{itemize}
  \item Based on the CryptoNote protocol described by a pseudonymous entity
    \textbf{Nickolas van Saberhagen}.
  \item Created by a pseudonymous individual \textbf{thankful\_for\_today} who
    forked the codebase of CryptoNote-based coin Bytecoin and launched the
    network.
  \item Similar to Bitcoin, has no central governing body.
  \item Very simple transaction structure with no scripting system.
  \item ASIC-resistant Proof-of-Work algorithm.
  \item Variable block size, short period between blocks (2 minutes).
  \item Protocol upgrades are performed via \textbf{planned hard forks}, which
    is only feasible because the network is comparatively small.
  \end{itemize}
\end{frame}

\begin{frame}
  \frametitle{Case Study: Monero 2/2}
  \begin{itemize}
  \item The privacy of Monero is achieved by concealing the following
    information:
    \begin{itemize}
    \item the sender of the money via the \textbf{ring signatures},
    \item the receiver of the money via the \textbf{stealth addresses},
    \item the amount sent via the \textbf{Confidential Transactions} system that
      encrypts the amounts in transaction but allows to check that the amount
      spend is greater than the amount received.
    \end{itemize}
  \item Monero is very low-use blockchain system and is therefore dangerous for
    serious use-cases.
  \end{itemize}
\end{frame}

\begin{frame}
  \frametitle{Founder Influence}
  \begin{itemize}
  \item Bitcoin and Monero are arguable the only interesting cryptocurrencies
    whose creators are no longer engaging with the community and have no
    influence over it.
  \item Satoshi Nakamoto never appeared online since 2011.
  \item Some people believe Satoshi disappeared to prevent any entity, including
    themselves, from manipulating the community using their name, identity,
    sentimental role, or the amount of bitcoin they presumably own.
  \item Satoshi's disappearance is considered a ``gift'' to Bitcoin ecosystem,
    as he/she removed himself/herself from the discussion to avoid undermining
    the decentralization of the system via \textbf{founder influence} in its
    formative years.
  \item Founder (Foundation) influence persuaded Ethereum community to accept a
    hard-fork in 2016 and lose credibility as a decentralized finance system.
  \end{itemize}
\end{frame}


\begin{frame}
  \frametitle{Conclusions}
  \begin{itemize}
  \item Bitcoin community is very conservative regarding the protocol changes,
    and network safety and stability are considered the most important goals.
  \item Features that Bitcoin lacks can often be implemented as layer-2
    solutions.
  \item Transaction privacy is one of the first priority goals (after safety and
    stability).
  \item \textbf{Is store of value Bitcoin's niche or will it become a single
      global payment system as well?}
  \item \textbf{Do we need other cryptocurrencies?}
  \end{itemize}
\end{frame}

\begin{frame}
  \frametitle{The End}
  \begin{center}
    Thank you!
  \end{center}
\end{frame}

\end{document}

%%% Local Variables:
%%% mode: latex
%%% TeX-master: t
%%% End:
