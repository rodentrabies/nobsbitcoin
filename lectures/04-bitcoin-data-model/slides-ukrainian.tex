\documentclass{beamer}


\usepackage{amsmath}
\usepackage[style=alphabetic,url=true]{biblatex}
\usepackage{environ}
\usepackage{geometry}
\usepackage{graphicx}
\usepackage{tikz}
\usepackage[T2A]{fontenc}
\usepackage[utf8]{inputenc}
\usepackage[cache=false]{minted}
\usepackage{amsmath}
\usepackage{amsfonts}
\usepackage{amssymb}
\usepackage{calrsfs}


% \usetheme{Bergen}

\usecolortheme{beaver}

\setbeamertemplate{itemize item}[circle]
\setbeamertemplate{itemize subitem}{--}
\addtobeamertemplate{navigation symbols}{}{
  \usebeamerfont{footline}%
  \usebeamercolor[fg]{footline}%
  \hspace{1em}%
  \insertframenumber/\inserttotalframenumber
}
\graphicspath{ {./graphics/} }
\setminted[Python]{
  fontsize=\tiny
}
\BeforeBeginEnvironment{minted}{\medskip}
\AfterEndEnvironment{minted}{\medskip}



\title{
  Біткоїн та криптовалютні технології \\
  Лекція 4: Модель даних Біткоїна
}

\author{Юрій Жикін}
\date{11 березня, 2024}

\begin{document}

\frame{\titlepage}

\begin{frame}
  \frametitle{Біткоїн-мережа}
  \begin{itemize}
  \item Біткоїн-мережа - це комп'ютерна мережа, що забезпечує у кожного учасника
    однакову копію бази даних з транзакціями, яка має спеціальну структуру
    (ланцюг блоків).
  \item ``Однакова копія'' означає, що у кожного учасника однаковий порядок 
    записів про транзакції.
  \end{itemize}
  \includegraphics[width=\textwidth]{network}
\end{frame}

\begin{frame}
  \frametitle{Модель даних Біткоїна}
  \begin{small}
    \begin{itemize}
    \item Біткоїн-транзакція - це запис у базі даних, яку підтримує
      Біткоїн-мережа, про те, від кого, кому і скільки переказано біткоїнів.
    \item \textbf{Ланцюг блоків} (або \textbf{часовий ланцюг}) - це розподілена,
      високо надлишкова база даних з транзакціями, що надійно гарантує
      \textit{існування, правильність і порядок} транзакцій.
    \item Кожен \textbf{Біткоїн-блок} - це підписана ``сторінка'' з записами про
      транзакції, побудована таким чином, щоб можна було легко перевірити
      правильність підпису та позицію транзакції в історії всіх транзакцій.
    \item Якщо запис про транзакцію потрапив до бази даних, ми можемо бути впевнені,
      що транзакція
      \begin{itemize}
      \item гарантовано існує і є правильно сконструйованою,
      \item строго слідує чи передує іншим транзакціям.
      \end{itemize}
    \end{itemize}
  \end{small}
\end{frame}

\begin{frame}
  \frametitle{Транзакція}
  \begin{itemize}
  \item \textbf{Транзакція}
    \begin{itemize}
    \item \textbf{версія}
    \item \textbf{входи}
    \item \textbf{виходи}
    \item \textbf{``свідки''}
    \item \textbf{час блокування}
    \end{itemize}
  \item \textbf{Входи} - список транзакційних входів - посилання на виходи інших
    транзакцій, які ``знищуються'' даною транзакцією.
  \item \textbf{Виходи} - список щойно створених виходів, які вказують, куди
    переводяться всі біткоїни з виходів, на які посилаються входи.
  \item \textbf{Час блокування} накладає обмеження на момент у часі, коли
    транзакція може бути включена в базу даних.
  \end{itemize}
\end{frame}

\begin{frame}[fragile]
  \frametitle{Ідентифікатор транзакції}
  \begin{itemize}
  \item \textbf{Ідентифікатор транзакції} не є частиною структури транзакції,
    натомість він обчислюється з бінарного представлення самої транзакції:
    $$TXID = SHA256(SHA256(TX_{binary}))$$
  \item $TXID$ - це послідовність з 32-х байтів, яка зазвичай представляється як
    64-символьний рядок у 16-му кодуванні:
    \begin{minted}{Python}
      169e1e83e930853391bc6f35f605c6754cfead57cf8387639d3b4096c54f18f4
    \end{minted}
  \end{itemize}
\end{frame}

\begin{frame}
  \frametitle{Транзакційний вихід}
  \begin{itemize}
  \item \textbf{Вихід}
    \begin{itemize}
    \item \textbf{кількість}
    \item \textbf{програма блокування}
    \end{itemize}
  \item \textbf{Кількість} - це кількість біткоїнів у даному виході, подана як
    ціле число, що означає кількість найменших одиниць, на які поділяється
    біткоїн, ``сатоші'' (1 BTC = $10^8$ сатоші).
  \item \textbf{Програма блокування} - обчислювальна задача (зазвичай ``надай
    правильний цифровий підпис''), яка повинна бути вирішена для того, щоб мати
    змогу використати даний вихід.
  \end{itemize}
\end{frame}

\begin{frame}
  \frametitle{Транзакційний вхід}
  \begin{itemize}
  \item \textbf{Вхід}
    \begin{itemize}
    \item \textbf{ідентифікатор попередньої транзакції}
    \item \textbf{індекс виходу у попередній транзакції}
    \item \textbf{програма розблокування}
    \end{itemize}
  \item Транзакційний вхід - це посилання на транзакційний вихід, який
    ``знищується'' у даній транзакції, а також програма розблокування.
  \item \textbf{Ідентифікатор попередньої транзакції} - TXID транзакції, що
    створила вихід.
  \item \textbf{Індекс у попередній транзакції} (\textbf{VOUT}) - індекс виходу
    у списку виходів в попередній транзакції.
  \item \textbf{Програма розблокування} - рішення задачі, сформульованої у
    програмі блокування цього виходу, предсталене, як програма мовою Bitcoin
    Script.
  \end{itemize}
\end{frame}

\begin{frame}
  \frametitle{Транзакційний ``свідок''}
  \begin{itemize}
  \item \textbf{Свідок} - це додаткова структура в Біткоїн-транзакції, яка була
    впроваджена у зміні до протоколу під назвою ``Відділений свідок'' (англ. \textbf{SegWit} -
    \textbf{Seg}regated \textbf{Wit}ness) 2017 року, як перший крок у
    довготривалому плані щодо вдосконалення безпеки, пропускної здатності та
    гнучкості Біткоїна.
  \item \textbf{Свідок} дозволяє зберігати складні \textit{скрипти розблокування}
    (рішення для \textit{скриптів блокування}, які складають значну частину всіх
    даних у транзакції.
  \end{itemize}
\end{frame}

\begin{frame}
  \frametitle{UTXO 1/2}
  \begin{itemize}
  \item Всі біткоїни, які існують у системі, представлені так званою
    \textbf{множиною невикористаних транзакційних виходів} (\textbf{UTXOs}) -
    множиною записів $(amount, owner)$, які не були використані як входи у
    жодній іншій транзакції, і це можна довести.
  \item Кожна звичайна Біткоїн-транзакція знищує певну кількість існуючих UTXO і
    створює певну кількість нових UTXO.
  \item Ми кажемо, що учасник Біткоїн-системи ``має'' біткоїн, якщо множина UTXO
    містить виходи, для яких цей учасник може створити \textbf{програму
      розблокування}, чим авторизує транзакцію, яка переведе цей біткоїн у
    власність якогось іншого учасника.
  \end{itemize}
\end{frame}

\begin{frame}
  \frametitle{UTXO 2/2}
  \begin{center}
    \includegraphics[width=\textwidth]{tx}
  \end{center}
\end{frame}

\begin{frame}[fragile]
  \frametitle{Комісія за транзакцію}
  \begin{itemize}
  \item \textbf{Комісія за транзакцію} - це різниця між сумарною кількість
    біткоїнів у виходах, що знищуються цією транзакцією, та сумарною кількістю
    біткоїнів у виходах, що створюються цією транзакцією:
    $$TxFee = \sum_{i=1}^n InputAmount(txin_i) - \sum_{j=1}^m Amount(txout_j),$$
  \end{itemize}
\end{frame}

\begin{frame}
  \frametitle{Блок}
  \begin{itemize}
  \item \textbf{Блок}
    \begin{itemize}
    \item \textbf{заголовок}
    \item \textbf{транзакції}
    \end{itemize}
  \item \textbf{Заголовок} - це структура, що містить метадані про блок і всі
    компоненти, необхідні для роботи системи ``доказу виконаної роботи''.
  \item \textbf{Транзакції} - це впорядкований список транзакцій, що входять у
    даний блок.
  \end{itemize}
\end{frame}

\begin{frame}
  \frametitle{Заголовок блока 1/4}
  \begin{small}
    \begin{itemize}
    \item \textbf{Заголовок}
      \begin{itemize}
      \item \textbf{версія}
      \item \textbf{ідентифікатор попереднього блока}
      \item \textbf{корінь мерклевого дерева транзакції}
      \item \textbf{час створення блока}
      \item \textbf{задача ``доказу виконаної роботи''}
      \item \textbf{одноразовий ключ ``доказу виконаної роботи''}
      \end{itemize}
    \item \textbf{Ідентифікатор блока} (або \textbf{хеш блока}) обчислюється
      аналогічно до ідентифікатора транзакції:
      $$BlockID = SHA256(SHA256(BlockHeader_{binary}))$$
    \item \textbf{Ідентифікатор попереднього блока} - це компонент, що забезпечує
      утворення ``ланцюга'' з блоків (звідки й термін ``ланцюг блоків''
      (англ. blockchain): кожен наступний блок посилається на попередній блок, і
      зміна будь-якої інформації в будь-якому блоці змінює хеш даного блока, а
      також \textbf{хеші всіх блоків, що слідують за ним} завдяки лавинній
      властивості криптографічних хеш-функцій.
    \end{itemize}
  \end{small}
\end{frame}

\begin{frame}
  \frametitle{Заголовок блока 2/4}
  \begin{small}
    \begin{itemize}
    \item \textbf{задача ``доказу виконаної роботи''} - це спеціальним чином
      закодоване число, яке є ціллю алгоритму ``доказу виконаної роботи'':
      ідентифікатор (хеш) блока повинен бути мешним за це число:
      $$ BlockID = SHA256(SHA256(BlockHeader)) < Target$$
    \item \textbf{Задача} переобчислюється кожних 2016 блоків (приблизно 2 тижні)
      для того, щоб підлаштувати складність знаходження рішень для нових блоків
      під вимогу, що середній період між блоками на кожному проміжку у 2016 блоків
      має становити 600 секунд (10 хвилин).
    \item Переобчислення значення задачі називається \textbf{коригуванням
        складності}: якщо середній час за останній 2016 блоків менший за 10
      хвилин, потрібно збільшити складність, обравши менше число-ціль, і навпаки.
    \end{itemize}
  \end{small}
\end{frame}

\begin{frame}
  \frametitle{Заголовок блока 3/4}
  \begin{itemize}
  \item \textbf{Корінь мерклевого дерева транзакцій} - корінь мерклевого дерева
    - 32-байтна послідовність, яка криптографічно містить всі транзакції в
    даному блоці і фіксує їх порядок:
  \end{itemize}
  \begin{center}
    \includegraphics[width=0.8\textwidth]{merkletree}
  \end{center}
\end{frame}

\begin{frame}
  \frametitle{Заголовок блока 4/4}
  \begin{itemize}
  \item Ідентифікатори попередніх блоків зв'язують блоки та транзакції, що в них
    містяться, у лінійну послідовність завдяки криптографічним ``зобов'язанням''.
  \item Зміна лише одного біта в будь-якій транзакції повністю змінює корінь
    мерклевого дерева, що змінює ідентифікатор блока, що в свою чергу змінює
    ідентифікатор наступного блока, і так далі.
  \end{itemize}
  \begin{center}
    \includegraphics[width=0.8\textwidth]{block-chain}
  \end{center}
\end{frame}

\begin{frame}
  \frametitle{Майнінг}
  \begin{small}
    \begin{itemize}
    \item \textbf{Майнінг} (з англ. ``видобування'' блоків) - це процес обчислення
      рішення задачі ``доказу виконаної роботи'' для нових блоків.
    \item Майнер
      \begin{itemize}
      \item обирає певну кількість транзакцій з множини непідтверджених транзакцій,
      \item будує Мерклеве дерево і використовує корінь цього дерева та хеш
        попереднього (``найвищого'') відомого йому блока у заголовку нового блока.
      \item здійснює пошук шляхом ``грубої сили'' (повний перебір варіантів)
        рішення ``доказу виконаної роботи''
        $$SHA256(SHA256(BlockHeader)) < Target$$
      \item якщо рішення знайдено до того, як майнер дізнається, що хтось інший
        знайшов його раніше (отримає блок, що має той же блок як попередній),
        він публікує новий блок у мережу і сподівається, що він буде прийнятий, як
        новий найвищий блок.
      \end{itemize}
    \end{itemize}
  \end{small}
\end{frame}

\begin{frame}
  \frametitle{Генеруюча транзакція і події поділу}
  \begin{small}
    \begin{itemize}
    \item Для того, щоб залучити майнерів до роботи на мережу, Біткоїн-протокол
      дозволяє їм додати першою транзакцією у блоці спеціальну транзакцію.
    \item Ця транзакція називається \textbf{генеруючою транзакцією}
      (англ. \textbf{coinbase}) і не має входів, лише виходи, кількість біткоїнів
      у яких встановлена протоколом, таким чином генеруючи нові біткоїни ``з
      повітря''.
    \item Окрім цього, майнер має право додати до нових біткоїнів сумарну різницю
      між входами та виходами всіх транзакцій у блоці (\textit{комісію за транзакції}).
    \item \textbf{Подія поділу навпіл} - для того, щоб гарантувати обмежену
      кількість біткоїнів в системі, кількість нових біткоїнів, що генеруються у
      блоці, почалась з 50 біткоїнів, зменшується вдвічі кожних 210000 блоків
      (приблизно кожних 4 роки), і рано чи пізно досягне 0 (орієнтовно через 120
      років), після чого кількість біткоїнів у системі перестане збільшуватись.
    \end{itemize}
  \end{small}
\end{frame}

\begin{frame}
  \frametitle{Корисні ресурси}
  \begin{itemize}
  \item Learn me a Bitcoin by Greg Walker - ресурс, що містить багато цікавої
    інформації про технічні деталі роботи Біткоїн-протоколу
    \begin{itemize}
    \item https://learnmeabitcoin.com/
    \end{itemize}
  \end{itemize}
\end{frame}

\begin{frame}
  \frametitle{Кінець}
  \begin{center}
    Дякую за увагу!
  \end{center}
\end{frame}

\end{document}

%%% Local Variables:
%%% mode: latex
%%% TeX-master: t
%%% End:
